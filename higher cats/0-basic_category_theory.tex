
\begin{definition}
    Given $F : \cat{C} \to \cat{D}$ and $x : \cat{D}$ we can define the slice category over $F$ as the pullback:
    \[\begin{tikzcd}
        \cat{C}_{/F} \ar[r] \ar[d] & \cat{D}_{/x} \ar[d] \\
        \cat{C} \ar[r] & \cat{D}
    \end{tikzcd}\]
    Explicitly objects are given as pairs $c : \cat{C}$ and morphisms $Fc \to x$.
\end{definition}

\begin{example}
    Over the Yoneda embedding $\yo : \cat{C} \to \hat{\cat{C}}$ and $F : \hat{\cat{C}}$ we have an element of 
    $\cat{C}_{/F}$ is a pair of $c : \cat{C}$ and a morphism $\yo_c \to F$, that is an element $F(c)$. 
\end{example}

\begin{lemma}
    Given $F : \cat{C}^\op \to \Set$ the natural map $\colim_{c \in \cat{C}_{/F}} \yo_c \to F$
    is an equivalence
\end{lemma}
\begin{proof}
    The natural map is given by choosing, for each $c : \cat{C}_{/F}$, the specified map $\yo_c \to F$.

    We have equivalences, for each presheaf $G$
    \[\hom_{\hat{\cat{C}}}( \colim_{c : \cat{C}_{/F}} \yo_c , G) 
    \cong \lim_{c \in \cat{C}_{/F}} \hom(\yo_c, G) \cong \lim_{c \in \cat{C}_{/F}} G(c) \simeq \hom(F, G) \] 
    giving the natural equivalence by Yoneda.
    The last equivalence is quick to check by hand.
\end{proof}

\begin{definition}[Kan extension (over inclusions)]
    Given $i : \cat{C}_0 \subseteq \cat{C}$, the inclusion of a subcategory, and $\cat{D}$ a (co)complete category,
    the restriction functor $i^\ast : \hom(\cat{C}, \cat{D}) \to \hom(\cat{C}_0, \cat{D})$ has left and right adjoints,
    given by 
    \[ i_!(F)(x) = \colim_{c \in \cat{C}_0 / x} F(c) \]
    \[ i_\ast(F)(x) = \lim_{c \in \cat{C}_0} F(c) \]
\end{definition}



\subsection{Simplicial sets}

We define the category $\Delta$ with objects given by $[n] = \{0, \ldots, n\}$ and morphisms given by monotone maps.
We define the category $\sSet$ to be presheaves on $\Delta$.
We denote the yoneda embedding of $n$ to be $\Delta^n$. 
This gives ``expected'' geometric results, such as $\hom_{\sSet}(\Delta^n, X) = X_n$, the $n$-simplices of $X$ are given by morphisms of $n$-simplices.

\begin{lemma}
    Given a simplicial set $X$, we have 
    \[X \cong \colim_{ [n] \in \Delta/X} \Delta^n \] 
\end{lemma}
\begin{proof}
    This is exactly the fact that presheaves are colimits of representables.
\end{proof}

\begin{definition}
    For a simplicial set $X$ we define 
    \[ \pi_0^\Delta(X) = X_0 / \sim\]
    where $x ~ y$ iff there exists $f : X_1$ such that $d_0(f) = x$ and $d_1(f) = y$.
\end{definition}

\begin{exercise}
    Find examples of simplicial sets where $\sim$ is not symmetric, or not transitive.
\paragraph{Solution}
    For the first example we consider $\Delta^1$.
    This has non-degenerate $0$-simplices given by $\Delta^1([0]) = [0] \to [1]$ by the constant maps at $0$ and $1$.
    And it has non-denerate $1$-simplex given by the identity map $(0,1) : [1] \to [1]$.
    This has $d_0 (0, 1) = 1$ and $d_1 (0, 1) = 0$ so that $1 \sim 0$.
    But there is no $1$-simplex giving $0 \sim 1$ so the relation isn't symmetric.
    
    For a non-transitive example consider the spine $I^2 \subseteq \Delta^2$.
    This has non-degenerate $1$-simplices given by $(0,1)$ and $(1,2)$.
    Hence we have $0 \sim 1$ and $1 \sim 2$ but not $0 \sim 2$.
\end{exercise}

\begin{exercise}
    Let $X : \sSet$.
    Every simplex $x : X_n$ is uniquely determined by a surjection $\alpha : [n] \to [m$ and $y : X_m$ with $y$ non-degenerate and $\alpha^*(y) = x$.

    \paragraph{Solution}
    Note if $x$ is degenerate, then by definition there exists some surjection $\alpha$ and some $y$ such that $\alpha^*(y) = x$.
    Similarly if $x$ is non-degenerate then there also exists a surjection, namely $\id : [n] \to [n]$, and this satisfies $\id^*(x) = x$.

    We take the minimum $[m]$ such that there exists an $\alpha$ and a $y$.
    We claim such a $y$ is non-degenerate. If it were then we would have a surjection $[k] \to [m]$ with $k < m$, composing would give a smaller value.
    Hence $y$ is non-degenerate.

    We now claim this is unique. Suppose we had $\beta : [n] \to [k]$ surjective with $\beta^*(z) = x$.
    Then by assumption $m \leq k$.
    By surjectivity we can construct a map $[k] \to [m]$ which composing with $\beta$ gives $\alpha$.
    Since $z$ is not degenerate we deduce $k = m$ and $\beta$ is the identity.
\end{exercise}

\begin{exercise}
    Compute the limit and colimit of a simplicial set $X : \Delta^\op \to \Set$.

    \paragraph{Solution} Since the category $\Delta^\op$ has an initial object, we can compute that the limit of $X$ is precisely $X_0$.
    The colimit of $X$ is given by the quotient of $\bigsqcup X_m$ by the relation generated by $x \sim \sigma^*(x)$. 
    Note that all elements are related to an element of $X_0$ by the first vertex map. 
    And pairs of elements of $X_0$ are related exactly when there is an $f : X_1$ connecting them.
    So we have that the colimit of $X$ is precisely $\pi_0^\Delta(X)$.
\end{exercise}

\begin{definition}
    In the category of topological spaces, we define $\Delta^n_\mathrm{Top}$ to be the subset of $\bR^{n+1}$ of tuples who sum to $1$.
    This defines a cosimplicial object in $\mathrm{Top}$ by 
    \[ [n] \mapsto \Delta^n_{\mathrm{Top}} \]
    \[ f : [n] \to [m] \mapsto f_*\]
    where $f_*$ is the unique affine linear extension of the map sending vertices of $\Delta^n_{\mathrm{Top}}$ to their image under $f$.
    More precisely $ f_*(t_0, \ldots t_n) = (v_0, \ldots, v_m) $ where 
    $ v_m = \sum_{j \in f^{-1}(i)} t_j $
\end{definition}

\begin{definition}
    Given a topological space $X$, its singular simplicial set is given by 
    \[S(X) = [n] \mapsto \hom(\Delta^n_{\mathrm{Top}}, X)\].

    We have a functor in the opposite direction given by Kan extension of the map sending $\Delta^n \mapsto \Delta^n_{\mathrm{Top}}$.
    Explicitly
    \[ |X| = \colim_{\Delta^n \in \Delta_{/X}} \Delta^n_{\mathrm{Top}}\]

    We have $| \_ |$ is left adjoint to $S$.
\end{definition}

\begin{definition}
    \begin{enumerate}
        \item $\partial \Delta^n$ has $m$-cells given by non-surjective $[m] \to [n]$.
        \item Given $S \subseteq [n]$ the \textbf{horn} $\Lambda^n_S$ has $m$-cells given $f : [m] \to [n]$ so there is $i \in [n] \backslash S$ s.t. $i \notin \mathrm{im}(f)$.
        \item The \textbf{spine} $I^n$ has $m$-cells given by $[m] \to [n]$ with image $\{j\}$ or $\{j, j+1\}$.
    \end{enumerate}
\end{definition}

\begin{definition}
    We can form the subcategory $\Delta_{\leq n} \subseteq \Delta$.
    The natural adjunction given by kan extending the inclusion gives monads on simplicial sets called the skeleta and coskeleta.
    Specifically
    \[\mathrm{sk}_n(X) = i_! i^\ast X \]
    \[\mathrm{cosk}_n(X) = i_* i^\ast X \]
    We have $\mathrm{cosk}_n(X)_k = \hom(\mathrm{sk}_n(\Delta^k), X)$
\end{definition}


\begin{exercise}
    For all $n \geq 0$ we have a pushout
    \[\begin{tikzcd}
        \bigsqcup_{J_n} \partial \Delta^n \ar[r] \ar[d] & \mathrm{sk}_{n-1}(X) \ar[d] \\
        \bigsqcup_{J_n} \Delta^n \ar[r] & \mathrm{sk}_{n}(X)
    \end{tikzcd}\]    
    Where $J_n$ are the non-degenerate $n$-simplices.

    \paragraph{Solution}
    We will turn this into a Yoneda style gluing argument to make our lives easier.
    Let $Y$ be a simplicial set. We have to show that 
    \[\begin{tikzcd}
        \hom(\bigsqcup_{J_n} \partial \Delta^n, Y) & \hom(\mathrm{sk}_{n-1}(X), Y) \ar[l]  \\
        \hom(\bigsqcup_{J_n} \Delta^n, Y) \ar[u] & \hom(\mathrm{sk}_{n}(X), Y) \ar[u] \ar[l]
    \end{tikzcd}\]
    is a pullback.
    By using the adjunction, and the Yoneda lemma this diagram is equivalent to
    \[\begin{tikzcd}
        \prod_{J_n} \hom(\partial \Delta^n, Y) & \hom(i_{n-1} X, i_{n-1} Y) \ar[l] \\
        (Y_n)^{J_n} \ar[u] & \hom(i_n X, i_n Y) \ar[u] \ar[l]
    \end{tikzcd}\]

    Further we note that a map $\partial \Delta^n \to Y$ is determined by its values on the $n-1$ cells of $\partial \Delta^n$,
    realising this set as a subset of $\prod_{i = 0}^n Y_{n-1}$.
    
    Now we can explicitly describe the maps in this diagram.
    $\hom(i_n X, i_n Y)$ are given by natural transformations, collections of $f_i : X_i \to Y_i$ for $0 \leq i \leq n$.
    \begin{itemize}
        \item The map $\hom(i_n X, i_n Y) \to \hom(i_{n-1} X, i_{n-1} Y)$ sends $(f_0, \ldots, f_n)$ to $(f_0, \ldots, f_{n-1})$.
        \item The map $\hom(i_n X, i_n Y) \to (Y_n)^{J_n}$ sends a tuple $f$ to the map $\lambda \sigma . f_n(\sigma)$.
        \item The map from $(Y_n)^{J_n}$ sends $g$ to the map $\lambda (\sigma, i). d_i g(\sigma)$.
        \item The top map sends a tuple $f$ to the map $\lambda (\sigma, i) . f_{n-1}(d_i \sigma)$. 
    \end{itemize}
    Thus to check this diagram is a pullback we need to show the following.
    Suppose we have maps $f : \hom(i_{n-1} X , i_{n-1} Y)$ and $g : (Y_n)^{J_n}$ which satisfy for all $\sigma : J_n$ and $i$ that $d_i g(\sigma) = f_{n-1}(d_i \sigma)$.
    THen we must construct a unique map $\hom(i_n X, i_n Y)$ mapping to $f$ and $g$.
    We do this by constructing $f_n$.
    If $x : X_n$ then either $x \in J_n$ in which case $f_n(x) := g(x)$,
    Or $x$ is degenerate and $x = \alpha^*(y)$ for some $y : X_m$ with $m < n$.
    Then define $f_n(x) = \alpha^*(f_m(y))$.
    This is well defined, natural and unique.
\end{exercise}


\begin{definition}
    Given a category $\cC$, its nerve is the simplicial set given by 
    \[ [n] \mapsto \hom([n], \cC) \]
    That is chains of $n$ composable maps.
\end{definition}

\begin{definition}
    Given a group $G$ its classifying space is the geometric realisation of its nerve as a category with one object.
\end{definition}

\begin{definition}
    A \textbf{Kan complex} is a simplicial set $X$ such that for all Horns $\Delta^n_j$ there are solutions to lifting problems
    \[\begin{tikzcd}
        \Lambda^n_j \ar[d] \ar[r] & X \\
        \Delta^n \ar[ur, dashed]
    \end{tikzcd}\]
\end{definition}

\begin{exercise}
    Given a simplicial set $X$ show that for all $n \geq 0$ that $\mathrm{cosk}_n(X)$ is a Kan complex.
    Further show the natural map $X \to \mathrm{cosk}_n(X)$ induces isomorphisms on $\pi_k$ for $k < n$ and that $\pi_k(\mathrm{cosk}_n(X)) = 0$ for $k \geq n$.

    \paragraph{Solution} Suppose we have a map $\Lambda_j^m \to \mathrm{cosk}_n(X)$.
    By adjunction this is equivalently given by a map $\mathrm{sk}_n(\Lambda_j^m) \to X$.
    Note that for $m > n+1$ the map $\mathrm{sk_n}(\Lambda_j^m) \to \mathrm{sk_n}(\Delta^m)$ is just the Horn inclusion.
    Hence we have a lift.
    If $m \leq n$ then the inclusion becomes an isomorphism, so therefore has a lift.
    In the case $m = n +1$ we have $\mathrm{sk_n}(\Lambda_j^{n+1}) \to \mathrm{sk_n}(\Delta^{n+1})$ becomes the inclusion $\Lambda_j^{n+1} \to \partial \Delta^{n+1}$.
    We can fill against this by filling two a the whole simplex, and restricting.

    This same argument makes the isomorphism of homotopy groups clear.
\end{exercise}

\begin{example}
    The singular simplicial set of a topological space is a Kan complex.
    This follows quickly by adjunction, noting that a horn is a retract of $\Delta^n$ in topological spaces.
\end{example}

\begin{example}
    The nerve of a category is a Kan complex if and only if the category is a groupoid.
\end{example}

\begin{definition}
    Given $f, g : X \to Y$ maps of simplicial sets, we say they are homotopic if there exist a map $H : \Delta^1 \times X \to Y$ restricting to $f, g$ on boundary.
    We can define the simplicial homotopy groups as homotopy classes of maps $(\Delta^n, \partial \Delta^n) \to (X, x)$.
\end{definition}
