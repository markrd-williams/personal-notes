
\begin{definition}
    Given $F : \cat{C} \to \cat{D}$ and $x : \cat{D}$ we can define the slice category over $F$ as the pullback:
    \[\begin{tikzcd}
        \cat{C}_{/F} \ar[r] \ar[d] & \cat{D}_{/x} \ar[d] \\
        \cat{C} \ar[r] & \cat{D}
    \end{tikzcd}\]
    Explicitly objects are given as pairs $c : \cat{C}$ and morphisms $Fc \to x$.
\end{definition}

\begin{example}
    Over the Yoneda embedding $\yo : \cat{C} \to \hat{\cat{C}}$ and $F : \hat{\cat{C}}$ we have an element of 
    $\cat{C}_{/F}$ is a pair of $c : \cat{C}$ and a morphism $\yo_c \to F$, that is an element $F(c)$. 
\end{example}

\begin{lemma}
    Given $F : \cat{C}^\op \to \Set$ the natural map $\colim_{c \in \cat{C}_{/F}} \yo_c \to F$
    is an equivalence
\end{lemma}
\begin{proof}
    The natural map is given by choosing, for each $c : \cat{C}_{/F}$, the specified map $\yo_c \to F$.

    We have equivalences, for each presheaf $G$
    \[\hom_{\hat{\cat{C}}}( \colim_{c : \cat{C}_{/F}} \yo_c , G) 
    \cong \lim_{c \in \cat{C}_{/F}} \hom(\yo_c, G) \cong \lim_{c \in \cat{C}_{/F}} G(c) \simeq \hom(F, G) \] 
    giving the natural equivalence by Yoneda.
    The last equivalence is quick to check by hand.
\end{proof}

\begin{definition}[Kan extension (over inclusions)]
    Given $i : \cat{C}_0 \subseteq \cat{C}$, the inclusion of a subcategory, and $\cat{D}$ a (co)complete category,
    the restriction functor $i^\ast : \hom(\cat{C}, \cat{D}) \to \hom(\cat{C}_0, \cat{D})$ has left and right adjoints,
    given by 
    \[ i_!(F)(x) = \colim_{c \in \cat{C}_0 / x} F(c) \]
    \[ i_\ast(F)(x) = \lim_{c \in \cat{C}_0} F(c) \]
\end{definition}



\subsection{Simplicial sets}

We define the category $\Delta$ with objects given by $[n] = \{0, \ldots, n\}$ and morphisms given by monotone maps.
We define the category $\sSet$ to be presheaves on $\Delta$.
We denote the yoneda embedding of $n$ to be $\Delta^n$. 
This gives ``expected'' geometric results, such as $\hom_{\sSet}(\Delta^n, X) = X_n$, the $n$-simplices of $X$ are given by morphisms of $n$-simplices.

\begin{lemma}
    Given a simplicial set $X$, we have 
    \[X \cong \colim_{ [n] \in \Delta/X} \Delta^n \] 
\end{lemma}
\begin{proof}
    This is exactly the fact that presheaves are colimits of representables.
\end{proof}

\begin{definition}
    In the category of topological spaces, we define $\Delta^n_\mathrm{Top}$ to be the subset of $\bR^{n+1}$ of tuples who sum to $1$.
    This defines a cosimplicial object in $\mathrm{Top}$ by 
    \[ [n] \mapsto \Delta^n_{\mathrm{Top}} \]
    \[ f : [n] \to [m] \mapsto f_*\]
    where $f_*$ is the unique affine linear extension of the map sending vertices of $\Delta^n_{\mathrm{Top}}$ to their image under $f$.
    More precisely:
    \[ f_*(t_0, \ldots t_n) = (v_0, \ldots, v_m)\]
    where
    \[ v_m = \sum_{j \in f^{-1}(i)} t_j\] 
\end{definition}

\begin{definition}
    Given a topological space $X$, its singular simplicial set is given by 
    \[S(X) = [n] \mapsto \hom(\Delta^n_{\mathrm{Top}}, X)\].

    We have a functor in the opposite direction given by Kan extension of the map sending $\Delta^n \mapsto \Delta^n_{\mathrm{Top}}$.
    Explicitly
    \[ |X| = \colim_{\Delta^n \in \Delta_{/X}} \Delta^n_{\mathrm{Top}}\]

    We have $| \_ |$ is left adjoint to $S$.
\end{definition}

\begin{definition}
    \begin{enumerate}
        \item $\partial \Delta^n$ has $m$-cells given by non-surjective $[m] \to [n]$.
        \item Given $S \subseteq [n]$ the \textbf{horn} $\Lambda^n_S$ has $m$-cells given $f : [m] \to [n]$ so there is $i \in [n] \backslash S$ s.t. $i \notin \mathrm{im}(f)$.
        \item The \textbf{spine} $I^n$ has $m$-cells given by $[m] \to [n]$ with image $\{j\}$ or $\{j, j+1\}$.
    \end{enumerate}
\end{definition}

\begin{definition}
    We can form the subcategory $\Delta_{\leq n} \subseteq \Delta$.
    The natural adjunction given by kan extending the inclusion gives monads on simplicial sets called the skeleta and coskeleta.
    Specifically
    \[\mathrm{sk}_n(X) = i_! i^\ast X \]
    \[\mathrm{cosk}_n(X) = i_* i^\ast X \]
    We have $\mathrm{cosk}_n(X)_k = \hom(\mathrm{sk}_n(\Delta^k), X)$
\end{definition}

\begin{definition}
    Given a category $\cC$, its nerve is the simplicial set given by 
    \[ [n] \mapsto \hom([n], \cC) \]
    That is chains of $n$ composable maps.
\end{definition}

\begin{definition}
    Given a group $G$ its classifying space is the geometric realisation of its nerve as a category with one object.
\end{definition}

\begin{definition}
    A \textbf{Kan complex} is a simplicial set $X$ such that for all Horns $\Delta^n_j$ there are solutions to lifting problems
    \[\begin{tikzcd}
        \Delta^n_j \ar[d] \ar[r] & X \\
        \Delta^n \ar[ur, dashed]
    \end{tikzcd}\]
\end{definition}

\begin{example}
    The singular simplicial set of a topological space is a Kan complex.
    This follows quickly by adjunction, noting that a horn is a retract of $\Delta^n$ in topological spaces.
\end{example}

\begin{example}
    The nerve of a category is a Kan complex if and only if the category is a groupoid.
\end{example}

\begin{definition}
    Given $f, g : X \to Y$ maps of simplicial sets, we say they are homotopic if there exist a map $H : \Delta^1 \times X \to Y$ restricting to $f, g$ on boundary.
    We can define the simplicial homotopy groups as homotopy classes of maps $(\Delta^n, \partial \Delta^n) \to (X, x)$.
\end{definition}


